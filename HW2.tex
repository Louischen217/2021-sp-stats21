\documentclass[11pt]{article}

    \usepackage[breakable]{tcolorbox}
    \usepackage{parskip} % Stop auto-indenting (to mimic markdown behaviour)
    
    \usepackage{iftex}
    \ifPDFTeX
    	\usepackage[T1]{fontenc}
    	\usepackage{mathpazo}
    \else
    	\usepackage{fontspec}
    \fi

    % Basic figure setup, for now with no caption control since it's done
    % automatically by Pandoc (which extracts ![](path) syntax from Markdown).
    \usepackage{graphicx}
    % Maintain compatibility with old templates. Remove in nbconvert 6.0
    \let\Oldincludegraphics\includegraphics
    % Ensure that by default, figures have no caption (until we provide a
    % proper Figure object with a Caption API and a way to capture that
    % in the conversion process - todo).
    \usepackage{caption}
    \DeclareCaptionFormat{nocaption}{}
    \captionsetup{format=nocaption,aboveskip=0pt,belowskip=0pt}

    \usepackage{float}
    \floatplacement{figure}{H} % forces figures to be placed at the correct location
    \usepackage{xcolor} % Allow colors to be defined
    \usepackage{enumerate} % Needed for markdown enumerations to work
    \usepackage{geometry} % Used to adjust the document margins
    \usepackage{amsmath} % Equations
    \usepackage{amssymb} % Equations
    \usepackage{textcomp} % defines textquotesingle
    % Hack from http://tex.stackexchange.com/a/47451/13684:
    \AtBeginDocument{%
        \def\PYZsq{\textquotesingle}% Upright quotes in Pygmentized code
    }
    \usepackage{upquote} % Upright quotes for verbatim code
    \usepackage{eurosym} % defines \euro
    \usepackage[mathletters]{ucs} % Extended unicode (utf-8) support
    \usepackage{fancyvrb} % verbatim replacement that allows latex
    \usepackage{grffile} % extends the file name processing of package graphics 
                         % to support a larger range
    \makeatletter % fix for old versions of grffile with XeLaTeX
    \@ifpackagelater{grffile}{2019/11/01}
    {
      % Do nothing on new versions
    }
    {
      \def\Gread@@xetex#1{%
        \IfFileExists{"\Gin@base".bb}%
        {\Gread@eps{\Gin@base.bb}}%
        {\Gread@@xetex@aux#1}%
      }
    }
    \makeatother
    \usepackage[Export]{adjustbox} % Used to constrain images to a maximum size
    \adjustboxset{max size={0.9\linewidth}{0.9\paperheight}}

    % The hyperref package gives us a pdf with properly built
    % internal navigation ('pdf bookmarks' for the table of contents,
    % internal cross-reference links, web links for URLs, etc.)
    \usepackage{hyperref}
    % The default LaTeX title has an obnoxious amount of whitespace. By default,
    % titling removes some of it. It also provides customization options.
    \usepackage{titling}
    \usepackage{longtable} % longtable support required by pandoc >1.10
    \usepackage{booktabs}  % table support for pandoc > 1.12.2
    \usepackage[inline]{enumitem} % IRkernel/repr support (it uses the enumerate* environment)
    \usepackage[normalem]{ulem} % ulem is needed to support strikethroughs (\sout)
                                % normalem makes italics be italics, not underlines
    \usepackage{mathrsfs}
    

    
    % Colors for the hyperref package
    \definecolor{urlcolor}{rgb}{0,.145,.698}
    \definecolor{linkcolor}{rgb}{.71,0.21,0.01}
    \definecolor{citecolor}{rgb}{.12,.54,.11}

    % ANSI colors
    \definecolor{ansi-black}{HTML}{3E424D}
    \definecolor{ansi-black-intense}{HTML}{282C36}
    \definecolor{ansi-red}{HTML}{E75C58}
    \definecolor{ansi-red-intense}{HTML}{B22B31}
    \definecolor{ansi-green}{HTML}{00A250}
    \definecolor{ansi-green-intense}{HTML}{007427}
    \definecolor{ansi-yellow}{HTML}{DDB62B}
    \definecolor{ansi-yellow-intense}{HTML}{B27D12}
    \definecolor{ansi-blue}{HTML}{208FFB}
    \definecolor{ansi-blue-intense}{HTML}{0065CA}
    \definecolor{ansi-magenta}{HTML}{D160C4}
    \definecolor{ansi-magenta-intense}{HTML}{A03196}
    \definecolor{ansi-cyan}{HTML}{60C6C8}
    \definecolor{ansi-cyan-intense}{HTML}{258F8F}
    \definecolor{ansi-white}{HTML}{C5C1B4}
    \definecolor{ansi-white-intense}{HTML}{A1A6B2}
    \definecolor{ansi-default-inverse-fg}{HTML}{FFFFFF}
    \definecolor{ansi-default-inverse-bg}{HTML}{000000}

    % common color for the border for error outputs.
    \definecolor{outerrorbackground}{HTML}{FFDFDF}

    % commands and environments needed by pandoc snippets
    % extracted from the output of `pandoc -s`
    \providecommand{\tightlist}{%
      \setlength{\itemsep}{0pt}\setlength{\parskip}{0pt}}
    \DefineVerbatimEnvironment{Highlighting}{Verbatim}{commandchars=\\\{\}}
    % Add ',fontsize=\small' for more characters per line
    \newenvironment{Shaded}{}{}
    \newcommand{\KeywordTok}[1]{\textcolor[rgb]{0.00,0.44,0.13}{\textbf{{#1}}}}
    \newcommand{\DataTypeTok}[1]{\textcolor[rgb]{0.56,0.13,0.00}{{#1}}}
    \newcommand{\DecValTok}[1]{\textcolor[rgb]{0.25,0.63,0.44}{{#1}}}
    \newcommand{\BaseNTok}[1]{\textcolor[rgb]{0.25,0.63,0.44}{{#1}}}
    \newcommand{\FloatTok}[1]{\textcolor[rgb]{0.25,0.63,0.44}{{#1}}}
    \newcommand{\CharTok}[1]{\textcolor[rgb]{0.25,0.44,0.63}{{#1}}}
    \newcommand{\StringTok}[1]{\textcolor[rgb]{0.25,0.44,0.63}{{#1}}}
    \newcommand{\CommentTok}[1]{\textcolor[rgb]{0.38,0.63,0.69}{\textit{{#1}}}}
    \newcommand{\OtherTok}[1]{\textcolor[rgb]{0.00,0.44,0.13}{{#1}}}
    \newcommand{\AlertTok}[1]{\textcolor[rgb]{1.00,0.00,0.00}{\textbf{{#1}}}}
    \newcommand{\FunctionTok}[1]{\textcolor[rgb]{0.02,0.16,0.49}{{#1}}}
    \newcommand{\RegionMarkerTok}[1]{{#1}}
    \newcommand{\ErrorTok}[1]{\textcolor[rgb]{1.00,0.00,0.00}{\textbf{{#1}}}}
    \newcommand{\NormalTok}[1]{{#1}}
    
    % Additional commands for more recent versions of Pandoc
    \newcommand{\ConstantTok}[1]{\textcolor[rgb]{0.53,0.00,0.00}{{#1}}}
    \newcommand{\SpecialCharTok}[1]{\textcolor[rgb]{0.25,0.44,0.63}{{#1}}}
    \newcommand{\VerbatimStringTok}[1]{\textcolor[rgb]{0.25,0.44,0.63}{{#1}}}
    \newcommand{\SpecialStringTok}[1]{\textcolor[rgb]{0.73,0.40,0.53}{{#1}}}
    \newcommand{\ImportTok}[1]{{#1}}
    \newcommand{\DocumentationTok}[1]{\textcolor[rgb]{0.73,0.13,0.13}{\textit{{#1}}}}
    \newcommand{\AnnotationTok}[1]{\textcolor[rgb]{0.38,0.63,0.69}{\textbf{\textit{{#1}}}}}
    \newcommand{\CommentVarTok}[1]{\textcolor[rgb]{0.38,0.63,0.69}{\textbf{\textit{{#1}}}}}
    \newcommand{\VariableTok}[1]{\textcolor[rgb]{0.10,0.09,0.49}{{#1}}}
    \newcommand{\ControlFlowTok}[1]{\textcolor[rgb]{0.00,0.44,0.13}{\textbf{{#1}}}}
    \newcommand{\OperatorTok}[1]{\textcolor[rgb]{0.40,0.40,0.40}{{#1}}}
    \newcommand{\BuiltInTok}[1]{{#1}}
    \newcommand{\ExtensionTok}[1]{{#1}}
    \newcommand{\PreprocessorTok}[1]{\textcolor[rgb]{0.74,0.48,0.00}{{#1}}}
    \newcommand{\AttributeTok}[1]{\textcolor[rgb]{0.49,0.56,0.16}{{#1}}}
    \newcommand{\InformationTok}[1]{\textcolor[rgb]{0.38,0.63,0.69}{\textbf{\textit{{#1}}}}}
    \newcommand{\WarningTok}[1]{\textcolor[rgb]{0.38,0.63,0.69}{\textbf{\textit{{#1}}}}}
    
    
    % Define a nice break command that doesn't care if a line doesn't already
    % exist.
    \def\br{\hspace*{\fill} \\* }
    % Math Jax compatibility definitions
    \def\gt{>}
    \def\lt{<}
    \let\Oldtex\TeX
    \let\Oldlatex\LaTeX
    \renewcommand{\TeX}{\textrm{\Oldtex}}
    \renewcommand{\LaTeX}{\textrm{\Oldlatex}}
    % Document parameters
    % Document title
    \title{HW2}
    
    
    
    
    
% Pygments definitions
\makeatletter
\def\PY@reset{\let\PY@it=\relax \let\PY@bf=\relax%
    \let\PY@ul=\relax \let\PY@tc=\relax%
    \let\PY@bc=\relax \let\PY@ff=\relax}
\def\PY@tok#1{\csname PY@tok@#1\endcsname}
\def\PY@toks#1+{\ifx\relax#1\empty\else%
    \PY@tok{#1}\expandafter\PY@toks\fi}
\def\PY@do#1{\PY@bc{\PY@tc{\PY@ul{%
    \PY@it{\PY@bf{\PY@ff{#1}}}}}}}
\def\PY#1#2{\PY@reset\PY@toks#1+\relax+\PY@do{#2}}

\expandafter\def\csname PY@tok@w\endcsname{\def\PY@tc##1{\textcolor[rgb]{0.73,0.73,0.73}{##1}}}
\expandafter\def\csname PY@tok@c\endcsname{\let\PY@it=\textit\def\PY@tc##1{\textcolor[rgb]{0.25,0.50,0.50}{##1}}}
\expandafter\def\csname PY@tok@cp\endcsname{\def\PY@tc##1{\textcolor[rgb]{0.74,0.48,0.00}{##1}}}
\expandafter\def\csname PY@tok@k\endcsname{\let\PY@bf=\textbf\def\PY@tc##1{\textcolor[rgb]{0.00,0.50,0.00}{##1}}}
\expandafter\def\csname PY@tok@kp\endcsname{\def\PY@tc##1{\textcolor[rgb]{0.00,0.50,0.00}{##1}}}
\expandafter\def\csname PY@tok@kt\endcsname{\def\PY@tc##1{\textcolor[rgb]{0.69,0.00,0.25}{##1}}}
\expandafter\def\csname PY@tok@o\endcsname{\def\PY@tc##1{\textcolor[rgb]{0.40,0.40,0.40}{##1}}}
\expandafter\def\csname PY@tok@ow\endcsname{\let\PY@bf=\textbf\def\PY@tc##1{\textcolor[rgb]{0.67,0.13,1.00}{##1}}}
\expandafter\def\csname PY@tok@nb\endcsname{\def\PY@tc##1{\textcolor[rgb]{0.00,0.50,0.00}{##1}}}
\expandafter\def\csname PY@tok@nf\endcsname{\def\PY@tc##1{\textcolor[rgb]{0.00,0.00,1.00}{##1}}}
\expandafter\def\csname PY@tok@nc\endcsname{\let\PY@bf=\textbf\def\PY@tc##1{\textcolor[rgb]{0.00,0.00,1.00}{##1}}}
\expandafter\def\csname PY@tok@nn\endcsname{\let\PY@bf=\textbf\def\PY@tc##1{\textcolor[rgb]{0.00,0.00,1.00}{##1}}}
\expandafter\def\csname PY@tok@ne\endcsname{\let\PY@bf=\textbf\def\PY@tc##1{\textcolor[rgb]{0.82,0.25,0.23}{##1}}}
\expandafter\def\csname PY@tok@nv\endcsname{\def\PY@tc##1{\textcolor[rgb]{0.10,0.09,0.49}{##1}}}
\expandafter\def\csname PY@tok@no\endcsname{\def\PY@tc##1{\textcolor[rgb]{0.53,0.00,0.00}{##1}}}
\expandafter\def\csname PY@tok@nl\endcsname{\def\PY@tc##1{\textcolor[rgb]{0.63,0.63,0.00}{##1}}}
\expandafter\def\csname PY@tok@ni\endcsname{\let\PY@bf=\textbf\def\PY@tc##1{\textcolor[rgb]{0.60,0.60,0.60}{##1}}}
\expandafter\def\csname PY@tok@na\endcsname{\def\PY@tc##1{\textcolor[rgb]{0.49,0.56,0.16}{##1}}}
\expandafter\def\csname PY@tok@nt\endcsname{\let\PY@bf=\textbf\def\PY@tc##1{\textcolor[rgb]{0.00,0.50,0.00}{##1}}}
\expandafter\def\csname PY@tok@nd\endcsname{\def\PY@tc##1{\textcolor[rgb]{0.67,0.13,1.00}{##1}}}
\expandafter\def\csname PY@tok@s\endcsname{\def\PY@tc##1{\textcolor[rgb]{0.73,0.13,0.13}{##1}}}
\expandafter\def\csname PY@tok@sd\endcsname{\let\PY@it=\textit\def\PY@tc##1{\textcolor[rgb]{0.73,0.13,0.13}{##1}}}
\expandafter\def\csname PY@tok@si\endcsname{\let\PY@bf=\textbf\def\PY@tc##1{\textcolor[rgb]{0.73,0.40,0.53}{##1}}}
\expandafter\def\csname PY@tok@se\endcsname{\let\PY@bf=\textbf\def\PY@tc##1{\textcolor[rgb]{0.73,0.40,0.13}{##1}}}
\expandafter\def\csname PY@tok@sr\endcsname{\def\PY@tc##1{\textcolor[rgb]{0.73,0.40,0.53}{##1}}}
\expandafter\def\csname PY@tok@ss\endcsname{\def\PY@tc##1{\textcolor[rgb]{0.10,0.09,0.49}{##1}}}
\expandafter\def\csname PY@tok@sx\endcsname{\def\PY@tc##1{\textcolor[rgb]{0.00,0.50,0.00}{##1}}}
\expandafter\def\csname PY@tok@m\endcsname{\def\PY@tc##1{\textcolor[rgb]{0.40,0.40,0.40}{##1}}}
\expandafter\def\csname PY@tok@gh\endcsname{\let\PY@bf=\textbf\def\PY@tc##1{\textcolor[rgb]{0.00,0.00,0.50}{##1}}}
\expandafter\def\csname PY@tok@gu\endcsname{\let\PY@bf=\textbf\def\PY@tc##1{\textcolor[rgb]{0.50,0.00,0.50}{##1}}}
\expandafter\def\csname PY@tok@gd\endcsname{\def\PY@tc##1{\textcolor[rgb]{0.63,0.00,0.00}{##1}}}
\expandafter\def\csname PY@tok@gi\endcsname{\def\PY@tc##1{\textcolor[rgb]{0.00,0.63,0.00}{##1}}}
\expandafter\def\csname PY@tok@gr\endcsname{\def\PY@tc##1{\textcolor[rgb]{1.00,0.00,0.00}{##1}}}
\expandafter\def\csname PY@tok@ge\endcsname{\let\PY@it=\textit}
\expandafter\def\csname PY@tok@gs\endcsname{\let\PY@bf=\textbf}
\expandafter\def\csname PY@tok@gp\endcsname{\let\PY@bf=\textbf\def\PY@tc##1{\textcolor[rgb]{0.00,0.00,0.50}{##1}}}
\expandafter\def\csname PY@tok@go\endcsname{\def\PY@tc##1{\textcolor[rgb]{0.53,0.53,0.53}{##1}}}
\expandafter\def\csname PY@tok@gt\endcsname{\def\PY@tc##1{\textcolor[rgb]{0.00,0.27,0.87}{##1}}}
\expandafter\def\csname PY@tok@err\endcsname{\def\PY@bc##1{\setlength{\fboxsep}{0pt}\fcolorbox[rgb]{1.00,0.00,0.00}{1,1,1}{\strut ##1}}}
\expandafter\def\csname PY@tok@kc\endcsname{\let\PY@bf=\textbf\def\PY@tc##1{\textcolor[rgb]{0.00,0.50,0.00}{##1}}}
\expandafter\def\csname PY@tok@kd\endcsname{\let\PY@bf=\textbf\def\PY@tc##1{\textcolor[rgb]{0.00,0.50,0.00}{##1}}}
\expandafter\def\csname PY@tok@kn\endcsname{\let\PY@bf=\textbf\def\PY@tc##1{\textcolor[rgb]{0.00,0.50,0.00}{##1}}}
\expandafter\def\csname PY@tok@kr\endcsname{\let\PY@bf=\textbf\def\PY@tc##1{\textcolor[rgb]{0.00,0.50,0.00}{##1}}}
\expandafter\def\csname PY@tok@bp\endcsname{\def\PY@tc##1{\textcolor[rgb]{0.00,0.50,0.00}{##1}}}
\expandafter\def\csname PY@tok@fm\endcsname{\def\PY@tc##1{\textcolor[rgb]{0.00,0.00,1.00}{##1}}}
\expandafter\def\csname PY@tok@vc\endcsname{\def\PY@tc##1{\textcolor[rgb]{0.10,0.09,0.49}{##1}}}
\expandafter\def\csname PY@tok@vg\endcsname{\def\PY@tc##1{\textcolor[rgb]{0.10,0.09,0.49}{##1}}}
\expandafter\def\csname PY@tok@vi\endcsname{\def\PY@tc##1{\textcolor[rgb]{0.10,0.09,0.49}{##1}}}
\expandafter\def\csname PY@tok@vm\endcsname{\def\PY@tc##1{\textcolor[rgb]{0.10,0.09,0.49}{##1}}}
\expandafter\def\csname PY@tok@sa\endcsname{\def\PY@tc##1{\textcolor[rgb]{0.73,0.13,0.13}{##1}}}
\expandafter\def\csname PY@tok@sb\endcsname{\def\PY@tc##1{\textcolor[rgb]{0.73,0.13,0.13}{##1}}}
\expandafter\def\csname PY@tok@sc\endcsname{\def\PY@tc##1{\textcolor[rgb]{0.73,0.13,0.13}{##1}}}
\expandafter\def\csname PY@tok@dl\endcsname{\def\PY@tc##1{\textcolor[rgb]{0.73,0.13,0.13}{##1}}}
\expandafter\def\csname PY@tok@s2\endcsname{\def\PY@tc##1{\textcolor[rgb]{0.73,0.13,0.13}{##1}}}
\expandafter\def\csname PY@tok@sh\endcsname{\def\PY@tc##1{\textcolor[rgb]{0.73,0.13,0.13}{##1}}}
\expandafter\def\csname PY@tok@s1\endcsname{\def\PY@tc##1{\textcolor[rgb]{0.73,0.13,0.13}{##1}}}
\expandafter\def\csname PY@tok@mb\endcsname{\def\PY@tc##1{\textcolor[rgb]{0.40,0.40,0.40}{##1}}}
\expandafter\def\csname PY@tok@mf\endcsname{\def\PY@tc##1{\textcolor[rgb]{0.40,0.40,0.40}{##1}}}
\expandafter\def\csname PY@tok@mh\endcsname{\def\PY@tc##1{\textcolor[rgb]{0.40,0.40,0.40}{##1}}}
\expandafter\def\csname PY@tok@mi\endcsname{\def\PY@tc##1{\textcolor[rgb]{0.40,0.40,0.40}{##1}}}
\expandafter\def\csname PY@tok@il\endcsname{\def\PY@tc##1{\textcolor[rgb]{0.40,0.40,0.40}{##1}}}
\expandafter\def\csname PY@tok@mo\endcsname{\def\PY@tc##1{\textcolor[rgb]{0.40,0.40,0.40}{##1}}}
\expandafter\def\csname PY@tok@ch\endcsname{\let\PY@it=\textit\def\PY@tc##1{\textcolor[rgb]{0.25,0.50,0.50}{##1}}}
\expandafter\def\csname PY@tok@cm\endcsname{\let\PY@it=\textit\def\PY@tc##1{\textcolor[rgb]{0.25,0.50,0.50}{##1}}}
\expandafter\def\csname PY@tok@cpf\endcsname{\let\PY@it=\textit\def\PY@tc##1{\textcolor[rgb]{0.25,0.50,0.50}{##1}}}
\expandafter\def\csname PY@tok@c1\endcsname{\let\PY@it=\textit\def\PY@tc##1{\textcolor[rgb]{0.25,0.50,0.50}{##1}}}
\expandafter\def\csname PY@tok@cs\endcsname{\let\PY@it=\textit\def\PY@tc##1{\textcolor[rgb]{0.25,0.50,0.50}{##1}}}

\def\PYZbs{\char`\\}
\def\PYZus{\char`\_}
\def\PYZob{\char`\{}
\def\PYZcb{\char`\}}
\def\PYZca{\char`\^}
\def\PYZam{\char`\&}
\def\PYZlt{\char`\<}
\def\PYZgt{\char`\>}
\def\PYZsh{\char`\#}
\def\PYZpc{\char`\%}
\def\PYZdl{\char`\$}
\def\PYZhy{\char`\-}
\def\PYZsq{\char`\'}
\def\PYZdq{\char`\"}
\def\PYZti{\char`\~}
% for compatibility with earlier versions
\def\PYZat{@}
\def\PYZlb{[}
\def\PYZrb{]}
\makeatother


    % For linebreaks inside Verbatim environment from package fancyvrb. 
    \makeatletter
        \newbox\Wrappedcontinuationbox 
        \newbox\Wrappedvisiblespacebox 
        \newcommand*\Wrappedvisiblespace {\textcolor{red}{\textvisiblespace}} 
        \newcommand*\Wrappedcontinuationsymbol {\textcolor{red}{\llap{\tiny$\m@th\hookrightarrow$}}} 
        \newcommand*\Wrappedcontinuationindent {3ex } 
        \newcommand*\Wrappedafterbreak {\kern\Wrappedcontinuationindent\copy\Wrappedcontinuationbox} 
        % Take advantage of the already applied Pygments mark-up to insert 
        % potential linebreaks for TeX processing. 
        %        {, <, #, %, $, ' and ": go to next line. 
        %        _, }, ^, &, >, - and ~: stay at end of broken line. 
        % Use of \textquotesingle for straight quote. 
        \newcommand*\Wrappedbreaksatspecials {% 
            \def\PYGZus{\discretionary{\char`\_}{\Wrappedafterbreak}{\char`\_}}% 
            \def\PYGZob{\discretionary{}{\Wrappedafterbreak\char`\{}{\char`\{}}% 
            \def\PYGZcb{\discretionary{\char`\}}{\Wrappedafterbreak}{\char`\}}}% 
            \def\PYGZca{\discretionary{\char`\^}{\Wrappedafterbreak}{\char`\^}}% 
            \def\PYGZam{\discretionary{\char`\&}{\Wrappedafterbreak}{\char`\&}}% 
            \def\PYGZlt{\discretionary{}{\Wrappedafterbreak\char`\<}{\char`\<}}% 
            \def\PYGZgt{\discretionary{\char`\>}{\Wrappedafterbreak}{\char`\>}}% 
            \def\PYGZsh{\discretionary{}{\Wrappedafterbreak\char`\#}{\char`\#}}% 
            \def\PYGZpc{\discretionary{}{\Wrappedafterbreak\char`\%}{\char`\%}}% 
            \def\PYGZdl{\discretionary{}{\Wrappedafterbreak\char`\$}{\char`\$}}% 
            \def\PYGZhy{\discretionary{\char`\-}{\Wrappedafterbreak}{\char`\-}}% 
            \def\PYGZsq{\discretionary{}{\Wrappedafterbreak\textquotesingle}{\textquotesingle}}% 
            \def\PYGZdq{\discretionary{}{\Wrappedafterbreak\char`\"}{\char`\"}}% 
            \def\PYGZti{\discretionary{\char`\~}{\Wrappedafterbreak}{\char`\~}}% 
        } 
        % Some characters . , ; ? ! / are not pygmentized. 
        % This macro makes them "active" and they will insert potential linebreaks 
        \newcommand*\Wrappedbreaksatpunct {% 
            \lccode`\~`\.\lowercase{\def~}{\discretionary{\hbox{\char`\.}}{\Wrappedafterbreak}{\hbox{\char`\.}}}% 
            \lccode`\~`\,\lowercase{\def~}{\discretionary{\hbox{\char`\,}}{\Wrappedafterbreak}{\hbox{\char`\,}}}% 
            \lccode`\~`\;\lowercase{\def~}{\discretionary{\hbox{\char`\;}}{\Wrappedafterbreak}{\hbox{\char`\;}}}% 
            \lccode`\~`\:\lowercase{\def~}{\discretionary{\hbox{\char`\:}}{\Wrappedafterbreak}{\hbox{\char`\:}}}% 
            \lccode`\~`\?\lowercase{\def~}{\discretionary{\hbox{\char`\?}}{\Wrappedafterbreak}{\hbox{\char`\?}}}% 
            \lccode`\~`\!\lowercase{\def~}{\discretionary{\hbox{\char`\!}}{\Wrappedafterbreak}{\hbox{\char`\!}}}% 
            \lccode`\~`\/\lowercase{\def~}{\discretionary{\hbox{\char`\/}}{\Wrappedafterbreak}{\hbox{\char`\/}}}% 
            \catcode`\.\active
            \catcode`\,\active 
            \catcode`\;\active
            \catcode`\:\active
            \catcode`\?\active
            \catcode`\!\active
            \catcode`\/\active 
            \lccode`\~`\~ 	
        }
    \makeatother

    \let\OriginalVerbatim=\Verbatim
    \makeatletter
    \renewcommand{\Verbatim}[1][1]{%
        %\parskip\z@skip
        \sbox\Wrappedcontinuationbox {\Wrappedcontinuationsymbol}%
        \sbox\Wrappedvisiblespacebox {\FV@SetupFont\Wrappedvisiblespace}%
        \def\FancyVerbFormatLine ##1{\hsize\linewidth
            \vtop{\raggedright\hyphenpenalty\z@\exhyphenpenalty\z@
                \doublehyphendemerits\z@\finalhyphendemerits\z@
                \strut ##1\strut}%
        }%
        % If the linebreak is at a space, the latter will be displayed as visible
        % space at end of first line, and a continuation symbol starts next line.
        % Stretch/shrink are however usually zero for typewriter font.
        \def\FV@Space {%
            \nobreak\hskip\z@ plus\fontdimen3\font minus\fontdimen4\font
            \discretionary{\copy\Wrappedvisiblespacebox}{\Wrappedafterbreak}
            {\kern\fontdimen2\font}%
        }%
        
        % Allow breaks at special characters using \PYG... macros.
        \Wrappedbreaksatspecials
        % Breaks at punctuation characters . , ; ? ! and / need catcode=\active 	
        \OriginalVerbatim[#1,codes*=\Wrappedbreaksatpunct]%
    }
    \makeatother

    % Exact colors from NB
    \definecolor{incolor}{HTML}{303F9F}
    \definecolor{outcolor}{HTML}{D84315}
    \definecolor{cellborder}{HTML}{CFCFCF}
    \definecolor{cellbackground}{HTML}{F7F7F7}
    
    % prompt
    \makeatletter
    \newcommand{\boxspacing}{\kern\kvtcb@left@rule\kern\kvtcb@boxsep}
    \makeatother
    \newcommand{\prompt}[4]{
        {\ttfamily\llap{{\color{#2}[#3]:\hspace{3pt}#4}}\vspace{-\baselineskip}}
    }
    

    
    % Prevent overflowing lines due to hard-to-break entities
    \sloppy 
    % Setup hyperref package
    \hypersetup{
      breaklinks=true,  % so long urls are correctly broken across lines
      colorlinks=true,
      urlcolor=urlcolor,
      linkcolor=linkcolor,
      citecolor=citecolor,
      }
    % Slightly bigger margins than the latex defaults
    
    \geometry{verbose,tmargin=1in,bmargin=1in,lmargin=1in,rmargin=1in}
    
    

\begin{document}
    
    \maketitle
    
    

    
    \hypertarget{stats-21---hw-2}{%
\section{Stats 21 - HW 2}\label{stats-21---hw-2}}

    \hypertarget{ailong-chen}{%
\subsection{Ailong Chen}\label{ailong-chen}}

    The questions have been entered into this document. You will modify the
document by entering your code.

Make sure you run the cell so the requested output is visible. Download
the finished document as a PDF file. If you are unable to convert it to
a PDF, you can download it as an HTML file and then print to PDF.

\textbf{Homework is an opportunity to practice coding and to practice
problem solving. Doing exercises is where you will do most of your
learning.}

\textbf{Copying someone else's solutions takes away your learning
opportunities. It is also academic dishonesty.}

    \hypertarget{reading}{%
\subsection{Reading}\label{reading}}

\begin{itemize}
\tightlist
\item
  Think Python: Chapters 6 through 10
\end{itemize}

\textbf{Reading is important!} Keep up with the reading. I recommend
alternating between reading a chapter and then working on exercises.

Additional recommended reading:

\begin{itemize}
\tightlist
\item
  String methods documentation
  https://docs.python.org/3/library/stdtypes.html\#string-methods
\end{itemize}

    \hypertarget{textbook-chapter-5-problems}{%
\subsection{Textbook Chapter 5
Problems}\label{textbook-chapter-5-problems}}

    \hypertarget{exercise-5.1}{%
\subsection{Exercise 5.1}\label{exercise-5.1}}

    \begin{tcolorbox}[breakable, size=fbox, boxrule=1pt, pad at break*=1mm,colback=cellbackground, colframe=cellborder]
\prompt{In}{incolor}{1}{\boxspacing}
\begin{Verbatim}[commandchars=\\\{\}]
\PY{k+kn}{import} \PY{n+nn}{time}
\end{Verbatim}
\end{tcolorbox}

    \begin{tcolorbox}[breakable, size=fbox, boxrule=1pt, pad at break*=1mm,colback=cellbackground, colframe=cellborder]
\prompt{In}{incolor}{2}{\boxspacing}
\begin{Verbatim}[commandchars=\\\{\}]
\PY{n}{time}\PY{o}{.}\PY{n}{time}\PY{p}{(}\PY{p}{)}
\end{Verbatim}
\end{tcolorbox}

            \begin{tcolorbox}[breakable, size=fbox, boxrule=.5pt, pad at break*=1mm, opacityfill=0]
\prompt{Out}{outcolor}{2}{\boxspacing}
\begin{Verbatim}[commandchars=\\\{\}]
1618988802.7442102
\end{Verbatim}
\end{tcolorbox}
        
    Write a function \texttt{now()} that reads the current time and prints
out the time of day in hours, minutes, and seconds, plus the number of
days since the epoch. The function does not need to return a value, just
print output to the screen.

The result should look like:

``Current time is: 15:25:47. It has been 18370 days since the epoch.''

Use \texttt{int()} to drop decimal values. You do not need to try to
find the date with years and months.

Tip: build your function incrementally. Start by finding how many days
have passed since the epoch. (check your answer at the bottom of the
page: https://www.epochconverter.com/seconds-days-since-y0 ) From there
find how many hours, etc. Keep in mind the hours will be UTC time.

    \begin{tcolorbox}[breakable, size=fbox, boxrule=1pt, pad at break*=1mm,colback=cellbackground, colframe=cellborder]
\prompt{In}{incolor}{3}{\boxspacing}
\begin{Verbatim}[commandchars=\\\{\}]
\PY{k}{def} \PY{n+nf}{now}\PY{p}{(}\PY{p}{)}\PY{p}{:}
    \PY{n}{the\PYZus{}time} \PY{o}{=} \PY{n}{time}\PY{o}{.}\PY{n}{time}\PY{p}{(}\PY{p}{)}
    \PY{n}{days} \PY{o}{=}  \PY{n+nb}{int}\PY{p}{(}\PY{n}{the\PYZus{}time} \PY{o}{/}\PY{o}{/} \PY{p}{(}\PY{l+m+mi}{24} \PY{o}{*} \PY{l+m+mi}{3600}\PY{p}{)}\PY{p}{)}
    \PY{n}{hours} \PY{o}{=} \PY{n+nb}{int}\PY{p}{(}\PY{p}{(}\PY{n}{the\PYZus{}time} \PY{o}{\PYZhy{}} \PY{n}{days} \PY{o}{*} \PY{l+m+mi}{24} \PY{o}{*} \PY{l+m+mi}{3600}\PY{p}{)} \PY{o}{/}\PY{o}{/} \PY{l+m+mi}{3600}\PY{p}{)}
    \PY{n}{mins} \PY{o}{=}  \PY{n+nb}{int}\PY{p}{(}\PY{p}{(}\PY{n}{the\PYZus{}time} \PY{o}{\PYZhy{}} \PY{n}{days} \PY{o}{*} \PY{l+m+mi}{24} \PY{o}{*} \PY{l+m+mi}{3600} \PY{o}{\PYZhy{}} \PY{n}{hours} \PY{o}{*} \PY{l+m+mi}{3600}\PY{p}{)} \PY{o}{/}\PY{o}{/} \PY{l+m+mi}{60}\PY{p}{)}
    \PY{n}{secs} \PY{o}{=}  \PY{n+nb}{int}\PY{p}{(}\PY{p}{(}\PY{n}{the\PYZus{}time} \PY{o}{\PYZhy{}} \PY{n}{days} \PY{o}{*} \PY{n}{hours} \PY{o}{*} \PY{l+m+mi}{3600}\PY{p}{)} \PY{o}{\PYZpc{}} \PY{l+m+mi}{60}\PY{p}{)}
    \PY{n+nb}{print}\PY{p}{(}\PY{l+s+s2}{\PYZdq{}}\PY{l+s+s2}{Current time is: }\PY{l+s+s2}{\PYZdq{}} \PY{o}{+} \PY{n+nb}{str}\PY{p}{(}\PY{n}{hours}\PY{p}{)} \PY{o}{+} \PY{l+s+s2}{\PYZdq{}}\PY{l+s+s2}{:}\PY{l+s+s2}{\PYZdq{}} \PY{o}{+} \PY{n+nb}{str}\PY{p}{(}\PY{n}{mins}\PY{p}{)} \PY{o}{+} \PY{l+s+s2}{\PYZdq{}}\PY{l+s+s2}{:}\PY{l+s+s2}{\PYZdq{}} \PY{o}{+} \PY{n+nb}{str}\PY{p}{(}\PY{n}{secs}\PY{p}{)} \PY{o}{+} \PY{l+s+s2}{\PYZdq{}}\PY{l+s+s2}{. It has been }\PY{l+s+s2}{\PYZdq{}} \PY{o}{+} \PY{n+nb}{str}\PY{p}{(}\PY{n}{days}\PY{p}{)} \PY{o}{+} \PY{l+s+s2}{\PYZdq{}}\PY{l+s+s2}{ days since the epoch.}\PY{l+s+s2}{\PYZdq{}}\PY{p}{)}
\end{Verbatim}
\end{tcolorbox}

    \begin{tcolorbox}[breakable, size=fbox, boxrule=1pt, pad at break*=1mm,colback=cellbackground, colframe=cellborder]
\prompt{In}{incolor}{4}{\boxspacing}
\begin{Verbatim}[commandchars=\\\{\}]
\PY{n}{now}\PY{p}{(}\PY{p}{)}
\end{Verbatim}
\end{tcolorbox}

    \begin{Verbatim}[commandchars=\\\{\}]
Current time is: 7:6:42. It has been 18738 days since the epoch.
    \end{Verbatim}

    \hypertarget{textbook-chapter-6-problems}{%
\subsection{Textbook Chapter 6
Problems}\label{textbook-chapter-6-problems}}

    \hypertarget{exercise-6.2}{%
\subsection{Exercise 6.2}\label{exercise-6.2}}

The Ackermann function, A(m, n), is defined:

\[
A(m, n)= \left \{                 
    \begin{array}{ll}
        n+1 & \text{if } m = 0\\
        A(m-1, 1) & \text{if } m > 0 \text{ and } n = 0\\
        A(m-1, A(m, n-1)) & \text{if } m > 0 \text{ and } n > 0\\
    \end{array}
 \right.
\]

See http://en.wikipedia.org/wiki/Ackermann\_function . Write a function
named \texttt{ack} that evaluates the Ackermann function. Use your
function to evaluate a few test cases. Don't test with \(m \ge 4\) as it
grows very fast very quickly.

    \begin{tcolorbox}[breakable, size=fbox, boxrule=1pt, pad at break*=1mm,colback=cellbackground, colframe=cellborder]
\prompt{In}{incolor}{5}{\boxspacing}
\begin{Verbatim}[commandchars=\\\{\}]
\PY{k}{def} \PY{n+nf}{ack}\PY{p}{(}\PY{n}{m}\PY{p}{,}\PY{n}{n}\PY{p}{)}\PY{p}{:}
    \PY{k}{if} \PY{n}{m} \PY{o}{==} \PY{l+m+mi}{0}\PY{p}{:}
        \PY{k}{return} \PY{n}{n} \PY{o}{+} \PY{l+m+mi}{1}
    \PY{k}{else}\PY{p}{:}
        \PY{k}{if} \PY{n}{n} \PY{o}{==} \PY{l+m+mi}{0}\PY{p}{:}
            \PY{k}{return} \PY{n}{ack}\PY{p}{(}\PY{n}{m} \PY{o}{\PYZhy{}} \PY{l+m+mi}{1}\PY{p}{,} \PY{l+m+mi}{1}\PY{p}{)}
        \PY{k}{else}\PY{p}{:}
            \PY{k}{return} \PY{n}{ack}\PY{p}{(}\PY{n}{m} \PY{o}{\PYZhy{}} \PY{l+m+mi}{1}\PY{p}{,} \PY{n}{ack}\PY{p}{(}\PY{n}{m}\PY{p}{,} \PY{n}{n} \PY{o}{\PYZhy{}} \PY{l+m+mi}{1}\PY{p}{)}\PY{p}{)}
\end{Verbatim}
\end{tcolorbox}

    \begin{tcolorbox}[breakable, size=fbox, boxrule=1pt, pad at break*=1mm,colback=cellbackground, colframe=cellborder]
\prompt{In}{incolor}{6}{\boxspacing}
\begin{Verbatim}[commandchars=\\\{\}]
\PY{c+c1}{\PYZsh{} test case, should be 61}
\PY{n}{ack}\PY{p}{(}\PY{l+m+mi}{3}\PY{p}{,} \PY{l+m+mi}{3}\PY{p}{)}
\end{Verbatim}
\end{tcolorbox}

            \begin{tcolorbox}[breakable, size=fbox, boxrule=.5pt, pad at break*=1mm, opacityfill=0]
\prompt{Out}{outcolor}{6}{\boxspacing}
\begin{Verbatim}[commandchars=\\\{\}]
61
\end{Verbatim}
\end{tcolorbox}
        
    \begin{tcolorbox}[breakable, size=fbox, boxrule=1pt, pad at break*=1mm,colback=cellbackground, colframe=cellborder]
\prompt{In}{incolor}{7}{\boxspacing}
\begin{Verbatim}[commandchars=\\\{\}]
\PY{c+c1}{\PYZsh{} test case, should be 125}
\PY{n}{ack}\PY{p}{(}\PY{l+m+mi}{3}\PY{p}{,} \PY{l+m+mi}{4}\PY{p}{)}
\end{Verbatim}
\end{tcolorbox}

            \begin{tcolorbox}[breakable, size=fbox, boxrule=.5pt, pad at break*=1mm, opacityfill=0]
\prompt{Out}{outcolor}{7}{\boxspacing}
\begin{Verbatim}[commandchars=\\\{\}]
125
\end{Verbatim}
\end{tcolorbox}
        
    \hypertarget{exercise-6.4}{%
\subsection{Exercise 6.4}\label{exercise-6.4}}

A number, \texttt{a}, is a power of \texttt{b} if it is divisible by
\texttt{b} and \texttt{a/b} is a power of \texttt{b}. Write a function
called \texttt{is\_power} that takes parameters \texttt{a} and
\texttt{b} and returns \texttt{True} if a is a power of b. Note: you
will have to think about the base case.

    \begin{tcolorbox}[breakable, size=fbox, boxrule=1pt, pad at break*=1mm,colback=cellbackground, colframe=cellborder]
\prompt{In}{incolor}{8}{\boxspacing}
\begin{Verbatim}[commandchars=\\\{\}]
\PY{k}{def} \PY{n+nf}{is\PYZus{}power}\PY{p}{(}\PY{n}{a}\PY{p}{,} \PY{n}{b}\PY{p}{)}\PY{p}{:}
    \PY{k}{if} \PY{n}{b} \PY{o}{==} \PY{l+m+mi}{1}\PY{p}{:}
        \PY{k}{return} \PY{k+kc}{False}
    \PY{k}{elif} \PY{n}{a} \PY{o}{==} \PY{l+m+mi}{1}\PY{p}{:}
        \PY{k}{return} \PY{k+kc}{True}
    \PY{k}{else}\PY{p}{:}
        \PY{k}{if} \PY{n}{a} \PY{o}{\PYZpc{}} \PY{n}{b}  \PY{o}{==} \PY{l+m+mi}{0}\PY{p}{:}
            \PY{k}{return} \PY{n}{is\PYZus{}power}\PY{p}{(}\PY{n}{a} \PY{o}{/} \PY{n}{b}\PY{p}{,} \PY{n}{b}\PY{p}{)}
        \PY{k}{else}\PY{p}{:}
            \PY{k}{return} \PY{k+kc}{False}
        
\end{Verbatim}
\end{tcolorbox}

    \begin{tcolorbox}[breakable, size=fbox, boxrule=1pt, pad at break*=1mm,colback=cellbackground, colframe=cellborder]
\prompt{In}{incolor}{9}{\boxspacing}
\begin{Verbatim}[commandchars=\\\{\}]
\PY{n}{is\PYZus{}power}\PY{p}{(}\PY{l+m+mi}{1024}\PY{p}{,} \PY{l+m+mi}{2}\PY{p}{)}
\end{Verbatim}
\end{tcolorbox}

            \begin{tcolorbox}[breakable, size=fbox, boxrule=.5pt, pad at break*=1mm, opacityfill=0]
\prompt{Out}{outcolor}{9}{\boxspacing}
\begin{Verbatim}[commandchars=\\\{\}]
True
\end{Verbatim}
\end{tcolorbox}
        
    \begin{tcolorbox}[breakable, size=fbox, boxrule=1pt, pad at break*=1mm,colback=cellbackground, colframe=cellborder]
\prompt{In}{incolor}{10}{\boxspacing}
\begin{Verbatim}[commandchars=\\\{\}]
\PY{n}{is\PYZus{}power}\PY{p}{(}\PY{l+m+mi}{6561}\PY{p}{,} \PY{l+m+mi}{3}\PY{p}{)}
\end{Verbatim}
\end{tcolorbox}

            \begin{tcolorbox}[breakable, size=fbox, boxrule=.5pt, pad at break*=1mm, opacityfill=0]
\prompt{Out}{outcolor}{10}{\boxspacing}
\begin{Verbatim}[commandchars=\\\{\}]
True
\end{Verbatim}
\end{tcolorbox}
        
    \begin{tcolorbox}[breakable, size=fbox, boxrule=1pt, pad at break*=1mm,colback=cellbackground, colframe=cellborder]
\prompt{In}{incolor}{11}{\boxspacing}
\begin{Verbatim}[commandchars=\\\{\}]
\PY{n}{is\PYZus{}power}\PY{p}{(}\PY{l+m+mi}{4374}\PY{p}{,} \PY{l+m+mi}{3}\PY{p}{)}
\end{Verbatim}
\end{tcolorbox}

            \begin{tcolorbox}[breakable, size=fbox, boxrule=.5pt, pad at break*=1mm, opacityfill=0]
\prompt{Out}{outcolor}{11}{\boxspacing}
\begin{Verbatim}[commandchars=\\\{\}]
False
\end{Verbatim}
\end{tcolorbox}
        
    \begin{tcolorbox}[breakable, size=fbox, boxrule=1pt, pad at break*=1mm,colback=cellbackground, colframe=cellborder]
\prompt{In}{incolor}{12}{\boxspacing}
\begin{Verbatim}[commandchars=\\\{\}]
\PY{n}{is\PYZus{}power}\PY{p}{(}\PY{l+m+mi}{768}\PY{p}{,} \PY{l+m+mi}{2}\PY{p}{)}
\end{Verbatim}
\end{tcolorbox}

            \begin{tcolorbox}[breakable, size=fbox, boxrule=.5pt, pad at break*=1mm, opacityfill=0]
\prompt{Out}{outcolor}{12}{\boxspacing}
\begin{Verbatim}[commandchars=\\\{\}]
False
\end{Verbatim}
\end{tcolorbox}
        
    \hypertarget{exercise-6.5}{%
\subsection{Exercise 6.5}\label{exercise-6.5}}

The greatest common divisor (GCD) of a and b is the largest number that
divides both of them with no remainder.

One way to find the GCD of two numbers is based on the observation that
if \texttt{r} is the remainder when \texttt{a} is divided by \texttt{b},
then \texttt{gcd(a,\ b)\ =\ gcd(b,\ r)}.

As a base case, we can use \texttt{gcd(a,\ 0)\ =\ a}.

Write a function called \texttt{gcd} that takes parameters \texttt{a}
and \texttt{b} and returns their greatest common divisor.

    \begin{tcolorbox}[breakable, size=fbox, boxrule=1pt, pad at break*=1mm,colback=cellbackground, colframe=cellborder]
\prompt{In}{incolor}{13}{\boxspacing}
\begin{Verbatim}[commandchars=\\\{\}]
\PY{k}{def} \PY{n+nf}{gcd}\PY{p}{(}\PY{n}{a}\PY{p}{,} \PY{n}{b}\PY{p}{)}\PY{p}{:}
    \PY{k}{if} \PY{n}{b} \PY{o}{==} \PY{l+m+mi}{0}\PY{p}{:}
        \PY{k}{return} \PY{n}{a}
    \PY{k}{else}\PY{p}{:}
        \PY{k}{return} \PY{n}{gcd}\PY{p}{(}\PY{n}{b}\PY{p}{,} \PY{n}{a} \PY{o}{\PYZpc{}} \PY{n}{b}\PY{p}{)}
\end{Verbatim}
\end{tcolorbox}

    \begin{tcolorbox}[breakable, size=fbox, boxrule=1pt, pad at break*=1mm,colback=cellbackground, colframe=cellborder]
\prompt{In}{incolor}{14}{\boxspacing}
\begin{Verbatim}[commandchars=\\\{\}]
\PY{n}{gcd}\PY{p}{(}\PY{l+m+mi}{21}\PY{p}{,} \PY{l+m+mi}{7}\PY{p}{)}
\end{Verbatim}
\end{tcolorbox}

            \begin{tcolorbox}[breakable, size=fbox, boxrule=.5pt, pad at break*=1mm, opacityfill=0]
\prompt{Out}{outcolor}{14}{\boxspacing}
\begin{Verbatim}[commandchars=\\\{\}]
7
\end{Verbatim}
\end{tcolorbox}
        
    \begin{tcolorbox}[breakable, size=fbox, boxrule=1pt, pad at break*=1mm,colback=cellbackground, colframe=cellborder]
\prompt{In}{incolor}{15}{\boxspacing}
\begin{Verbatim}[commandchars=\\\{\}]
\PY{n}{gcd}\PY{p}{(}\PY{l+m+mi}{42}\PY{p}{,} \PY{l+m+mi}{28}\PY{p}{)}
\end{Verbatim}
\end{tcolorbox}

            \begin{tcolorbox}[breakable, size=fbox, boxrule=.5pt, pad at break*=1mm, opacityfill=0]
\prompt{Out}{outcolor}{15}{\boxspacing}
\begin{Verbatim}[commandchars=\\\{\}]
14
\end{Verbatim}
\end{tcolorbox}
        
    \begin{tcolorbox}[breakable, size=fbox, boxrule=1pt, pad at break*=1mm,colback=cellbackground, colframe=cellborder]
\prompt{In}{incolor}{16}{\boxspacing}
\begin{Verbatim}[commandchars=\\\{\}]
\PY{n}{gcd}\PY{p}{(}\PY{l+m+mi}{105}\PY{p}{,} \PY{l+m+mi}{140}\PY{p}{)}
\end{Verbatim}
\end{tcolorbox}

            \begin{tcolorbox}[breakable, size=fbox, boxrule=.5pt, pad at break*=1mm, opacityfill=0]
\prompt{Out}{outcolor}{16}{\boxspacing}
\begin{Verbatim}[commandchars=\\\{\}]
35
\end{Verbatim}
\end{tcolorbox}
        
    \hypertarget{textbook-chapter-7-problems}{%
\subsection{Textbook Chapter 7
Problems}\label{textbook-chapter-7-problems}}

    \hypertarget{exercise-7.1}{%
\subsection{Exercise 7.1}\label{exercise-7.1}}

Copy the loop from Section 7.5 on square roots and encapsulate it into a
function called \texttt{mysqrt()} that takes \texttt{a} as a parameter.
For a starting value \texttt{x} use \texttt{a/2}. It then iterates
through the code to estimate the square root of a value.

Write another function called \texttt{test\_square\_root(start,\ end)}
that will print out a table as shown in the textbook.

    \begin{tcolorbox}[breakable, size=fbox, boxrule=1pt, pad at break*=1mm,colback=cellbackground, colframe=cellborder]
\prompt{In}{incolor}{17}{\boxspacing}
\begin{Verbatim}[commandchars=\\\{\}]
\PY{c+c1}{\PYZsh{} write your code here}
\PY{k}{def} \PY{n+nf}{mysqrt}\PY{p}{(}\PY{n}{a}\PY{p}{)}\PY{p}{:}
    \PY{n}{x} \PY{o}{=} \PY{n}{a} \PY{o}{/} \PY{l+m+mi}{2}
    \PY{k}{while} \PY{k+kc}{True}\PY{p}{:}
        \PY{n}{y} \PY{o}{=} \PY{p}{(}\PY{n}{x} \PY{o}{+} \PY{n}{a}\PY{o}{/}\PY{n}{x}\PY{p}{)} \PY{o}{/} \PY{l+m+mi}{2}
        \PY{k}{if} \PY{n+nb}{abs}\PY{p}{(}\PY{n}{y}\PY{o}{\PYZhy{}}\PY{n}{x}\PY{p}{)} \PY{o}{\PYZlt{}} \PY{l+m+mf}{0.0000000001}\PY{p}{:}
            \PY{k}{break}
        \PY{n}{x} \PY{o}{=} \PY{n}{y}
    \PY{k}{return} \PY{n}{x}



\PY{k}{def} \PY{n+nf}{test\PYZus{}square\PYZus{}root}\PY{p}{(}\PY{n}{start}\PY{p}{,} \PY{n}{end}\PY{p}{)}\PY{p}{:}
    \PY{k+kn}{import} \PY{n+nn}{math}
    \PY{n+nb}{print}\PY{p}{(}\PY{l+s+s2}{\PYZdq{}}\PY{l+s+s2}{a}\PY{l+s+se}{\PYZbs{}t}\PY{l+s+s2}{ mysqrt(a)}\PY{l+s+se}{\PYZbs{}t}\PY{l+s+se}{\PYZbs{}t}\PY{l+s+se}{\PYZbs{}t}\PY{l+s+s2}{math.sqrt(a)}\PY{l+s+se}{\PYZbs{}t}\PY{l+s+se}{\PYZbs{}t}\PY{l+s+se}{\PYZbs{}t}\PY{l+s+s2}{ diff}\PY{l+s+s2}{\PYZdq{}}\PY{p}{)}
    \PY{n+nb}{print}\PY{p}{(}\PY{l+s+s2}{\PYZdq{}}\PY{l+s+s2}{\PYZhy{}}\PY{l+s+se}{\PYZbs{}t}\PY{l+s+s2}{\PYZhy{}\PYZhy{}\PYZhy{}\PYZhy{}\PYZhy{}\PYZhy{}\PYZhy{}\PYZhy{}\PYZhy{}\PYZhy{}\PYZhy{}\PYZhy{}\PYZhy{}\PYZhy{}\PYZhy{}\PYZhy{}\PYZhy{}\PYZhy{}}\PY{l+s+se}{\PYZbs{}t}\PY{l+s+se}{\PYZbs{}t}\PY{l+s+se}{\PYZbs{}t}\PY{l+s+s2}{\PYZhy{}\PYZhy{}\PYZhy{}\PYZhy{}\PYZhy{}\PYZhy{}\PYZhy{}\PYZhy{}\PYZhy{}\PYZhy{}\PYZhy{}\PYZhy{}\PYZhy{}\PYZhy{}\PYZhy{}\PYZhy{}\PYZhy{}\PYZhy{}}\PY{l+s+se}{\PYZbs{}t}\PY{l+s+se}{\PYZbs{}t}\PY{l+s+se}{\PYZbs{}t}\PY{l+s+s2}{\PYZhy{}\PYZhy{}\PYZhy{}\PYZhy{}\PYZhy{}\PYZhy{}\PYZhy{}\PYZhy{}\PYZhy{}\PYZhy{}\PYZhy{}\PYZhy{}\PYZhy{}\PYZhy{}\PYZhy{}\PYZhy{}\PYZhy{}\PYZhy{}}\PY{l+s+s2}{\PYZdq{}}\PY{p}{)}
    \PY{k}{for} \PY{n}{i} \PY{o+ow}{in} \PY{n+nb}{range}\PY{p}{(}\PY{n+nb}{int}\PY{p}{(}\PY{n}{start}\PY{p}{)}\PY{p}{,}\PY{n+nb}{int}\PY{p}{(}\PY{n}{end}\PY{p}{)} \PY{o}{+} \PY{l+m+mi}{1}\PY{p}{)}\PY{p}{:}
        \PY{n}{res\PYZus{}1} \PY{o}{=} \PY{n}{mysqrt}\PY{p}{(}\PY{n}{i}\PY{p}{)}
        \PY{n}{res\PYZus{}2} \PY{o}{=} \PY{n}{math}\PY{o}{.}\PY{n}{sqrt}\PY{p}{(}\PY{n}{i}\PY{p}{)}
        \PY{n+nb}{print}\PY{p}{(}\PY{n+nb}{str}\PY{p}{(}\PY{n}{i}\PY{p}{)} \PY{o}{+} \PY{n+nb}{str}\PY{p}{(}\PY{l+s+s2}{\PYZdq{}}\PY{l+s+se}{\PYZbs{}t}\PY{l+s+s2}{\PYZdq{}}\PY{p}{)} \PY{o}{+} \PY{n+nb}{str}\PY{p}{(}\PY{n}{res\PYZus{}1}\PY{p}{)} \PY{o}{+} \PY{n+nb}{str}\PY{p}{(}\PY{l+s+s1}{\PYZsq{}}\PY{l+s+se}{\PYZbs{}t}\PY{l+s+se}{\PYZbs{}t}\PY{l+s+se}{\PYZbs{}t}\PY{l+s+s1}{\PYZsq{}}\PY{p}{)}\PY{o}{+} \PY{n+nb}{str}\PY{p}{(}\PY{n}{res\PYZus{}2}\PY{p}{)} \PY{o}{+} \PY{n+nb}{str}\PY{p}{(}\PY{l+s+s1}{\PYZsq{}}\PY{l+s+se}{\PYZbs{}t}\PY{l+s+se}{\PYZbs{}t}\PY{l+s+se}{\PYZbs{}t}\PY{l+s+s1}{\PYZsq{}}\PY{p}{)} \PY{o}{+} \PY{n+nb}{str}\PY{p}{(}\PY{n+nb}{abs}\PY{p}{(}\PY{n}{res\PYZus{}1} \PY{o}{\PYZhy{}} \PY{n}{res\PYZus{}2}\PY{p}{)}\PY{p}{)}\PY{p}{)}
\end{Verbatim}
\end{tcolorbox}

    \begin{tcolorbox}[breakable, size=fbox, boxrule=1pt, pad at break*=1mm,colback=cellbackground, colframe=cellborder]
\prompt{In}{incolor}{18}{\boxspacing}
\begin{Verbatim}[commandchars=\\\{\}]
\PY{c+c1}{\PYZsh{} test code, do not modify:}
\PY{n}{test\PYZus{}square\PYZus{}root}\PY{p}{(}\PY{l+m+mf}{1.0}\PY{p}{,} \PY{l+m+mf}{9.0}\PY{p}{)}
\end{Verbatim}
\end{tcolorbox}

    \begin{Verbatim}[commandchars=\\\{\}]
a        mysqrt(a)                      math.sqrt(a)                     diff
-       ------------------                      ------------------
------------------
1       1.000000000000001                       1.0
1.1102230246251565e-15
2       1.4142135623746899                      1.4142135623730951
1.5947243525715749e-12
3       1.7320508075688772                      1.7320508075688772
0.0
4       2.0                     2.0                     0.0
5       2.23606797749979                        2.23606797749979
0.0
6       2.4494897427831788                      2.449489742783178
8.881784197001252e-16
7       2.6457513110646933                      2.6457513110645907
1.0258460747536446e-13
8       2.8284271247493797                      2.8284271247461903
3.1894487051431497e-12
9       3.0000000000393214                      3.0
3.9321434996963944e-11
    \end{Verbatim}

    \begin{tcolorbox}[breakable, size=fbox, boxrule=1pt, pad at break*=1mm,colback=cellbackground, colframe=cellborder]
\prompt{In}{incolor}{19}{\boxspacing}
\begin{Verbatim}[commandchars=\\\{\}]
\PY{n}{test\PYZus{}square\PYZus{}root}\PY{p}{(}\PY{l+m+mi}{30}\PY{p}{,} \PY{l+m+mi}{35}\PY{p}{)}
\end{Verbatim}
\end{tcolorbox}

    \begin{Verbatim}[commandchars=\\\{\}]
a        mysqrt(a)                      math.sqrt(a)                     diff
-       ------------------                      ------------------
------------------
30      5.477225575051661                       5.477225575051661
0.0
31      5.567764362830022                       5.5677643628300215
8.881784197001252e-16
32      5.65685424949238                        5.656854249492381
8.881784197001252e-16
33      5.744562646538029                       5.744562646538029
0.0
34      5.830951894845301                       5.830951894845301
0.0
35      5.916079783099616                       5.916079783099616
0.0
    \end{Verbatim}

    \hypertarget{textbook-chapter-9-problems}{%
\subsection{Textbook Chapter 9
Problems}\label{textbook-chapter-9-problems}}

    \hypertarget{exercise-9.1}{%
\subsection{Exercise 9.1}\label{exercise-9.1}}

Download this list of words: http://thinkpython2.com/code/words.txt

Write and run a script that reads \texttt{words.txt} and prints out only
the words with more than 20 characters (after stripping whitespace).

    \begin{tcolorbox}[breakable, size=fbox, boxrule=1pt, pad at break*=1mm,colback=cellbackground, colframe=cellborder]
\prompt{In}{incolor}{20}{\boxspacing}
\begin{Verbatim}[commandchars=\\\{\}]
\PY{n}{fin} \PY{o}{=} \PY{n+nb}{open}\PY{p}{(}\PY{l+s+s2}{\PYZdq{}}\PY{l+s+s2}{words.txt}\PY{l+s+s2}{\PYZdq{}}\PY{p}{)}
\PY{n+nb}{print}\PY{p}{(}\PY{l+s+s2}{\PYZdq{}}\PY{l+s+s2}{The word with more than 20 characters:}\PY{l+s+se}{\PYZbs{}n}\PY{l+s+s2}{\PYZdq{}}\PY{p}{)}
\PY{k}{for} \PY{n}{line} \PY{o+ow}{in} \PY{n}{fin}\PY{p}{:}
    \PY{k}{if} \PY{n+nb}{len}\PY{p}{(}\PY{n}{line}\PY{p}{)} \PY{o}{\PYZgt{}} \PY{l+m+mi}{20}\PY{p}{:}
        \PY{n+nb}{print}\PY{p}{(}\PY{n}{line}\PY{p}{)}
\end{Verbatim}
\end{tcolorbox}

    \begin{Verbatim}[commandchars=\\\{\}]
The word with more than 20 characters:

counterdemonstration

counterdemonstrations

counterdemonstrators

hyperaggressivenesses

hypersensitivenesses

microminiaturization

microminiaturizations

representativenesses

    \end{Verbatim}

    \hypertarget{exercise-9.2}{%
\subsection{Exercise 9.2}\label{exercise-9.2}}

Write a function called \texttt{has\_no\_e} that returns True if the
word doesn't have the letter e. You can use any of Pythons availble
string methods.

    \begin{tcolorbox}[breakable, size=fbox, boxrule=1pt, pad at break*=1mm,colback=cellbackground, colframe=cellborder]
\prompt{In}{incolor}{21}{\boxspacing}
\begin{Verbatim}[commandchars=\\\{\}]
\PY{k}{def} \PY{n+nf}{has\PYZus{}no\PYZus{}e}\PY{p}{(}\PY{n}{text}\PY{p}{)}\PY{p}{:}
    \PY{k}{return} \PY{l+s+s1}{\PYZsq{}}\PY{l+s+s1}{e}\PY{l+s+s1}{\PYZsq{}} \PY{o+ow}{not} \PY{o+ow}{in} \PY{n}{text}
\end{Verbatim}
\end{tcolorbox}

    \begin{tcolorbox}[breakable, size=fbox, boxrule=1pt, pad at break*=1mm,colback=cellbackground, colframe=cellborder]
\prompt{In}{incolor}{22}{\boxspacing}
\begin{Verbatim}[commandchars=\\\{\}]
\PY{n}{has\PYZus{}no\PYZus{}e}\PY{p}{(}\PY{l+s+s2}{\PYZdq{}}\PY{l+s+s2}{hello}\PY{l+s+s2}{\PYZdq{}}\PY{p}{)}
\end{Verbatim}
\end{tcolorbox}

            \begin{tcolorbox}[breakable, size=fbox, boxrule=.5pt, pad at break*=1mm, opacityfill=0]
\prompt{Out}{outcolor}{22}{\boxspacing}
\begin{Verbatim}[commandchars=\\\{\}]
False
\end{Verbatim}
\end{tcolorbox}
        
    \begin{tcolorbox}[breakable, size=fbox, boxrule=1pt, pad at break*=1mm,colback=cellbackground, colframe=cellborder]
\prompt{In}{incolor}{23}{\boxspacing}
\begin{Verbatim}[commandchars=\\\{\}]
\PY{n}{has\PYZus{}no\PYZus{}e}\PY{p}{(}\PY{l+s+s2}{\PYZdq{}}\PY{l+s+s2}{quit}\PY{l+s+s2}{\PYZdq{}}\PY{p}{)}
\end{Verbatim}
\end{tcolorbox}

            \begin{tcolorbox}[breakable, size=fbox, boxrule=.5pt, pad at break*=1mm, opacityfill=0]
\prompt{Out}{outcolor}{23}{\boxspacing}
\begin{Verbatim}[commandchars=\\\{\}]
True
\end{Verbatim}
\end{tcolorbox}
        
    With your function, write a script. The script should read the list of
words (\texttt{words.txt}), print out the number of words that do not
have the letter `e' and the proportion of words that do not have the
letter `e'

    \begin{tcolorbox}[breakable, size=fbox, boxrule=1pt, pad at break*=1mm,colback=cellbackground, colframe=cellborder]
\prompt{In}{incolor}{24}{\boxspacing}
\begin{Verbatim}[commandchars=\\\{\}]
\PY{n}{fin} \PY{o}{=} \PY{n+nb}{open}\PY{p}{(}\PY{l+s+s2}{\PYZdq{}}\PY{l+s+s2}{words.txt}\PY{l+s+s2}{\PYZdq{}}\PY{p}{)}
\PY{n}{num\PYZus{}words} \PY{o}{=} \PY{l+m+mi}{0}
\PY{n}{num\PYZus{}no\PYZus{}e} \PY{o}{=} \PY{l+m+mi}{0}
\PY{k}{for} \PY{n}{line} \PY{o+ow}{in} \PY{n}{fin}\PY{p}{:}
    \PY{n}{num\PYZus{}words} \PY{o}{=} \PY{l+m+mi}{1} \PY{o}{+} \PY{n}{num\PYZus{}words}
    \PY{k}{if} \PY{n}{has\PYZus{}no\PYZus{}e}\PY{p}{(}\PY{n}{line}\PY{p}{)}\PY{p}{:}
        \PY{n}{num\PYZus{}no\PYZus{}e} \PY{o}{=} \PY{l+m+mi}{1} \PY{o}{+} \PY{n}{num\PYZus{}no\PYZus{}e}
\PY{n+nb}{print}\PY{p}{(}\PY{l+s+s2}{\PYZdq{}}\PY{l+s+s2}{The number of words that do not have the letter }\PY{l+s+s2}{\PYZsq{}}\PY{l+s+s2}{e}\PY{l+s+s2}{\PYZsq{}}\PY{l+s+s2}{ is: }\PY{l+s+s2}{\PYZdq{}} \PY{o}{+} \PY{n+nb}{str}\PY{p}{(}\PY{n}{num\PYZus{}no\PYZus{}e}\PY{p}{)}\PY{p}{)}
\PY{n+nb}{print}\PY{p}{(}\PY{l+s+s2}{\PYZdq{}}\PY{l+s+s2}{The proportion of words that do not have the letter }\PY{l+s+s2}{\PYZsq{}}\PY{l+s+s2}{e}\PY{l+s+s2}{\PYZsq{}}\PY{l+s+s2}{ is: }\PY{l+s+s2}{\PYZdq{}} \PY{o}{+} \PY{n+nb}{str}\PY{p}{(}\PY{n}{num\PYZus{}no\PYZus{}e} \PY{o}{/} \PY{n}{num\PYZus{}words} \PY{o}{*} \PY{l+m+mi}{100}\PY{p}{)} \PY{o}{+} \PY{l+s+s2}{\PYZdq{}}\PY{l+s+s2}{\PYZpc{}}\PY{l+s+s2}{\PYZdq{}}\PY{p}{)}
        
    
\end{Verbatim}
\end{tcolorbox}

    \begin{Verbatim}[commandchars=\\\{\}]
The number of words that do not have the letter 'e' is: 37641
The proportion of words that do not have the letter 'e' is: 33.07383423103621\%
    \end{Verbatim}

    \hypertarget{textbook-chapter-10-problems}{%
\subsection{Textbook Chapter 10
Problems}\label{textbook-chapter-10-problems}}

    \hypertarget{exercise-10.1}{%
\subsection{Exercise 10.1}\label{exercise-10.1}}

Write a function called \texttt{nested\_sum} that takes a list of lists
of integers and adds up the elements from all of the nested lists. For
example:

\begin{verbatim}
t = [[1, 2], [3], [4, 5, 6]]
nested_sum(t)
21
\end{verbatim}

You may want to build the function recursively in case there are many
levels of nested lists.

You can assume that all elements in any of the nested lists are numeric.

    \begin{tcolorbox}[breakable, size=fbox, boxrule=1pt, pad at break*=1mm,colback=cellbackground, colframe=cellborder]
\prompt{In}{incolor}{25}{\boxspacing}
\begin{Verbatim}[commandchars=\\\{\}]
\PY{k}{def} \PY{n+nf}{nested\PYZus{}sum}\PY{p}{(}\PY{n}{t}\PY{p}{)}\PY{p}{:}
    \PY{n+nb}{sum} \PY{o}{=} \PY{l+m+mi}{0}
    \PY{k}{for} \PY{n}{element} \PY{o+ow}{in} \PY{n}{t}\PY{p}{:}
        \PY{k}{if} \PY{n+nb}{type}\PY{p}{(}\PY{n}{element}\PY{p}{)} \PY{o}{==} \PY{n+nb}{list}\PY{p}{:}
            \PY{n+nb}{sum} \PY{o}{=} \PY{n+nb}{sum} \PY{o}{+} \PY{n}{nested\PYZus{}sum}\PY{p}{(}\PY{n}{element}\PY{p}{)} 
        \PY{k}{else}\PY{p}{:}
            \PY{n+nb}{sum} \PY{o}{=} \PY{n+nb}{sum} \PY{o}{+} \PY{n}{element}
    \PY{k}{return} \PY{n+nb}{sum}
            
\end{Verbatim}
\end{tcolorbox}

    \begin{tcolorbox}[breakable, size=fbox, boxrule=1pt, pad at break*=1mm,colback=cellbackground, colframe=cellborder]
\prompt{In}{incolor}{26}{\boxspacing}
\begin{Verbatim}[commandchars=\\\{\}]
\PY{n}{t} \PY{o}{=} \PY{p}{[}\PY{l+m+mi}{1}\PY{p}{,} \PY{l+m+mi}{2}\PY{p}{]}
\PY{n}{nested\PYZus{}sum}\PY{p}{(}\PY{n}{t}\PY{p}{)}
\end{Verbatim}
\end{tcolorbox}

            \begin{tcolorbox}[breakable, size=fbox, boxrule=.5pt, pad at break*=1mm, opacityfill=0]
\prompt{Out}{outcolor}{26}{\boxspacing}
\begin{Verbatim}[commandchars=\\\{\}]
3
\end{Verbatim}
\end{tcolorbox}
        
    \begin{tcolorbox}[breakable, size=fbox, boxrule=1pt, pad at break*=1mm,colback=cellbackground, colframe=cellborder]
\prompt{In}{incolor}{27}{\boxspacing}
\begin{Verbatim}[commandchars=\\\{\}]
\PY{n}{t} \PY{o}{=} \PY{p}{[}\PY{p}{[}\PY{l+m+mi}{1}\PY{p}{,} \PY{l+m+mi}{2}\PY{p}{]}\PY{p}{,} \PY{p}{[}\PY{l+m+mi}{3}\PY{p}{]}\PY{p}{,} \PY{p}{[}\PY{l+m+mi}{4}\PY{p}{,} \PY{l+m+mi}{5}\PY{p}{,} \PY{l+m+mi}{6}\PY{p}{]}\PY{p}{]}
\PY{n}{nested\PYZus{}sum}\PY{p}{(}\PY{n}{t}\PY{p}{)}
\end{Verbatim}
\end{tcolorbox}

            \begin{tcolorbox}[breakable, size=fbox, boxrule=.5pt, pad at break*=1mm, opacityfill=0]
\prompt{Out}{outcolor}{27}{\boxspacing}
\begin{Verbatim}[commandchars=\\\{\}]
21
\end{Verbatim}
\end{tcolorbox}
        
    \begin{tcolorbox}[breakable, size=fbox, boxrule=1pt, pad at break*=1mm,colback=cellbackground, colframe=cellborder]
\prompt{In}{incolor}{28}{\boxspacing}
\begin{Verbatim}[commandchars=\\\{\}]
\PY{n}{x} \PY{o}{=} \PY{p}{[}\PY{p}{[}\PY{l+m+mi}{1}\PY{p}{,} \PY{l+m+mi}{2}\PY{p}{,} \PY{p}{[}\PY{l+m+mi}{3}\PY{p}{]}\PY{p}{]}\PY{p}{,} \PY{l+m+mi}{4}\PY{p}{,} \PY{l+m+mi}{5}\PY{p}{,} \PY{l+m+mi}{6}\PY{p}{,} \PY{p}{[}\PY{l+m+mi}{7}\PY{p}{]}\PY{p}{,} \PY{l+m+mi}{8}\PY{p}{]}
\PY{n}{nested\PYZus{}sum}\PY{p}{(}\PY{n}{x}\PY{p}{)}
\end{Verbatim}
\end{tcolorbox}

            \begin{tcolorbox}[breakable, size=fbox, boxrule=.5pt, pad at break*=1mm, opacityfill=0]
\prompt{Out}{outcolor}{28}{\boxspacing}
\begin{Verbatim}[commandchars=\\\{\}]
36
\end{Verbatim}
\end{tcolorbox}
        
    \begin{tcolorbox}[breakable, size=fbox, boxrule=1pt, pad at break*=1mm,colback=cellbackground, colframe=cellborder]
\prompt{In}{incolor}{29}{\boxspacing}
\begin{Verbatim}[commandchars=\\\{\}]
\PY{n}{t} \PY{o}{=} \PY{p}{[}\PY{p}{[}\PY{p}{[}\PY{l+m+mi}{1}\PY{p}{,} \PY{l+m+mi}{2}\PY{p}{,} \PY{p}{[}\PY{l+m+mi}{3}\PY{p}{]}\PY{p}{]}\PY{p}{,} \PY{p}{[}\PY{l+m+mi}{4}\PY{p}{,} \PY{p}{[}\PY{l+m+mi}{5}\PY{p}{,} \PY{l+m+mi}{6}\PY{p}{,} \PY{p}{[}\PY{l+m+mi}{7}\PY{p}{]}\PY{p}{]}\PY{p}{,} \PY{l+m+mi}{8}\PY{p}{]}\PY{p}{]}\PY{p}{]}
\PY{n}{nested\PYZus{}sum}\PY{p}{(}\PY{n}{t}\PY{p}{)}
\end{Verbatim}
\end{tcolorbox}

            \begin{tcolorbox}[breakable, size=fbox, boxrule=.5pt, pad at break*=1mm, opacityfill=0]
\prompt{Out}{outcolor}{29}{\boxspacing}
\begin{Verbatim}[commandchars=\\\{\}]
36
\end{Verbatim}
\end{tcolorbox}
        
    \hypertarget{exercise-10.2}{%
\subsection{Exercise 10.2}\label{exercise-10.2}}

Write a function called \texttt{cumsum} that takes a list of numbers and
returns the cumulative sum; that is, a new list where the ith element is
the sum of the first i + 1 elements from the original list.

For example:

\begin{verbatim}
t = [1, 2, 3]
cumsum(t)
[1, 3, 6]
\end{verbatim}

You can assume that all elements in the lists are numeric and the list
does not contain nested lists.

    \begin{tcolorbox}[breakable, size=fbox, boxrule=1pt, pad at break*=1mm,colback=cellbackground, colframe=cellborder]
\prompt{In}{incolor}{30}{\boxspacing}
\begin{Verbatim}[commandchars=\\\{\}]
\PY{k}{def} \PY{n+nf}{cumsum}\PY{p}{(}\PY{n}{t}\PY{p}{)}\PY{p}{:}
    \PY{n}{new\PYZus{}list} \PY{o}{=} \PY{p}{[}\PY{p}{]}
    \PY{k}{for} \PY{n}{i} \PY{o+ow}{in} \PY{n+nb}{range}\PY{p}{(}\PY{n+nb}{len}\PY{p}{(}\PY{n}{t}\PY{p}{)}\PY{p}{)}\PY{p}{:}
        \PY{n}{new\PYZus{}list}\PY{o}{.}\PY{n}{append}\PY{p}{(}\PY{n+nb}{sum}\PY{p}{(}\PY{n}{t}\PY{p}{[}\PY{l+m+mi}{0}\PY{p}{:}\PY{n}{i}\PY{o}{+}\PY{l+m+mi}{1}\PY{p}{]}\PY{p}{)}\PY{p}{)}
    \PY{k}{return} \PY{n}{new\PYZus{}list}
\end{Verbatim}
\end{tcolorbox}

    \begin{tcolorbox}[breakable, size=fbox, boxrule=1pt, pad at break*=1mm,colback=cellbackground, colframe=cellborder]
\prompt{In}{incolor}{31}{\boxspacing}
\begin{Verbatim}[commandchars=\\\{\}]
\PY{n}{cumsum}\PY{p}{(}\PY{p}{[}\PY{l+m+mi}{1}\PY{p}{,} \PY{l+m+mi}{2}\PY{p}{,} \PY{l+m+mi}{3}\PY{p}{,} \PY{l+m+mi}{4}\PY{p}{]}\PY{p}{)}
\end{Verbatim}
\end{tcolorbox}

            \begin{tcolorbox}[breakable, size=fbox, boxrule=.5pt, pad at break*=1mm, opacityfill=0]
\prompt{Out}{outcolor}{31}{\boxspacing}
\begin{Verbatim}[commandchars=\\\{\}]
[1, 3, 6, 10]
\end{Verbatim}
\end{tcolorbox}
        
    \begin{tcolorbox}[breakable, size=fbox, boxrule=1pt, pad at break*=1mm,colback=cellbackground, colframe=cellborder]
\prompt{In}{incolor}{32}{\boxspacing}
\begin{Verbatim}[commandchars=\\\{\}]
\PY{n}{cumsum}\PY{p}{(}\PY{n+nb}{range}\PY{p}{(}\PY{l+m+mi}{12}\PY{p}{)}\PY{p}{)}
\end{Verbatim}
\end{tcolorbox}

            \begin{tcolorbox}[breakable, size=fbox, boxrule=.5pt, pad at break*=1mm, opacityfill=0]
\prompt{Out}{outcolor}{32}{\boxspacing}
\begin{Verbatim}[commandchars=\\\{\}]
[0, 1, 3, 6, 10, 15, 21, 28, 36, 45, 55, 66]
\end{Verbatim}
\end{tcolorbox}
        
    \hypertarget{exercise-10.6}{%
\subsection{Exercise 10.6}\label{exercise-10.6}}

Two words are anagrams if you can rearrange the letters from one to
spell the other. Write a function called \texttt{is\_anagram} that takes
two strings and returns \texttt{True} if they are anagrams.

You can remove spaces and convert to lowercase using
\texttt{string.replace("\ ","").lower()}

    \begin{tcolorbox}[breakable, size=fbox, boxrule=1pt, pad at break*=1mm,colback=cellbackground, colframe=cellborder]
\prompt{In}{incolor}{33}{\boxspacing}
\begin{Verbatim}[commandchars=\\\{\}]
\PY{k}{def} \PY{n+nf}{is\PYZus{}anagram}\PY{p}{(}\PY{n}{word1}\PY{p}{,} \PY{n}{word2}\PY{p}{)}\PY{p}{:}
    \PY{k}{if} \PY{n+nb}{len}\PY{p}{(}\PY{n}{word1}\PY{o}{.}\PY{n}{replace}\PY{p}{(}\PY{l+s+s2}{\PYZdq{}}\PY{l+s+s2}{ }\PY{l+s+s2}{\PYZdq{}}\PY{p}{,}\PY{l+s+s2}{\PYZdq{}}\PY{l+s+s2}{\PYZdq{}}\PY{p}{)}\PY{o}{.}\PY{n}{lower}\PY{p}{(}\PY{p}{)}\PY{p}{)} \PY{o}{!=} \PY{n+nb}{len}\PY{p}{(}\PY{n}{word2}\PY{o}{.}\PY{n}{replace}\PY{p}{(}\PY{l+s+s2}{\PYZdq{}}\PY{l+s+s2}{ }\PY{l+s+s2}{\PYZdq{}}\PY{p}{,}\PY{l+s+s2}{\PYZdq{}}\PY{l+s+s2}{\PYZdq{}}\PY{p}{)}\PY{o}{.}\PY{n}{lower}\PY{p}{(}\PY{p}{)}\PY{p}{)}\PY{p}{:}
        \PY{k}{return} \PY{k+kc}{False}
    \PY{k}{else}\PY{p}{:}
        \PY{k}{if} \PY{n+nb}{list}\PY{p}{(}\PY{n}{word1}\PY{p}{)}\PY{o}{.}\PY{n}{sort}\PY{p}{(}\PY{p}{)} \PY{o}{==} \PY{n+nb}{list}\PY{p}{(}\PY{n}{word2}\PY{p}{)}\PY{o}{.}\PY{n}{sort}\PY{p}{(}\PY{p}{)}\PY{p}{:}
            \PY{k}{return} \PY{k+kc}{True}
        \PY{k}{else}\PY{p}{:}
            \PY{k}{return} \PY{k+kc}{False}
        
\end{Verbatim}
\end{tcolorbox}

    \begin{tcolorbox}[breakable, size=fbox, boxrule=1pt, pad at break*=1mm,colback=cellbackground, colframe=cellborder]
\prompt{In}{incolor}{34}{\boxspacing}
\begin{Verbatim}[commandchars=\\\{\}]
\PY{n}{is\PYZus{}anagram}\PY{p}{(}\PY{l+s+s2}{\PYZdq{}}\PY{l+s+s2}{hello}\PY{l+s+s2}{\PYZdq{}}\PY{p}{,} \PY{l+s+s2}{\PYZdq{}}\PY{l+s+s2}{o hell}\PY{l+s+s2}{\PYZdq{}}\PY{p}{)}
\end{Verbatim}
\end{tcolorbox}

            \begin{tcolorbox}[breakable, size=fbox, boxrule=.5pt, pad at break*=1mm, opacityfill=0]
\prompt{Out}{outcolor}{34}{\boxspacing}
\begin{Verbatim}[commandchars=\\\{\}]
True
\end{Verbatim}
\end{tcolorbox}
        
    \begin{tcolorbox}[breakable, size=fbox, boxrule=1pt, pad at break*=1mm,colback=cellbackground, colframe=cellborder]
\prompt{In}{incolor}{35}{\boxspacing}
\begin{Verbatim}[commandchars=\\\{\}]
\PY{n}{is\PYZus{}anagram}\PY{p}{(}\PY{l+s+s2}{\PYZdq{}}\PY{l+s+s2}{dormitory}\PY{l+s+s2}{\PYZdq{}} \PY{p}{,} \PY{l+s+s2}{\PYZdq{}}\PY{l+s+s2}{dirty room}\PY{l+s+s2}{\PYZdq{}}\PY{p}{)}
\end{Verbatim}
\end{tcolorbox}

            \begin{tcolorbox}[breakable, size=fbox, boxrule=.5pt, pad at break*=1mm, opacityfill=0]
\prompt{Out}{outcolor}{35}{\boxspacing}
\begin{Verbatim}[commandchars=\\\{\}]
True
\end{Verbatim}
\end{tcolorbox}
        
    \begin{tcolorbox}[breakable, size=fbox, boxrule=1pt, pad at break*=1mm,colback=cellbackground, colframe=cellborder]
\prompt{In}{incolor}{36}{\boxspacing}
\begin{Verbatim}[commandchars=\\\{\}]
\PY{n}{is\PYZus{}anagram}\PY{p}{(}\PY{l+s+s2}{\PYZdq{}}\PY{l+s+s2}{dormitory}\PY{l+s+s2}{\PYZdq{}} \PY{p}{,} \PY{l+s+s2}{\PYZdq{}}\PY{l+s+s2}{dirty rooms}\PY{l+s+s2}{\PYZdq{}}\PY{p}{)}
\end{Verbatim}
\end{tcolorbox}

            \begin{tcolorbox}[breakable, size=fbox, boxrule=.5pt, pad at break*=1mm, opacityfill=0]
\prompt{Out}{outcolor}{36}{\boxspacing}
\begin{Verbatim}[commandchars=\\\{\}]
False
\end{Verbatim}
\end{tcolorbox}
        
    \begin{tcolorbox}[breakable, size=fbox, boxrule=1pt, pad at break*=1mm,colback=cellbackground, colframe=cellborder]
\prompt{In}{incolor}{37}{\boxspacing}
\begin{Verbatim}[commandchars=\\\{\}]
\PY{n}{is\PYZus{}anagram}\PY{p}{(}\PY{l+s+s2}{\PYZdq{}}\PY{l+s+s2}{astronomers}\PY{l+s+s2}{\PYZdq{}} \PY{p}{,} \PY{l+s+s2}{\PYZdq{}}\PY{l+s+s2}{moon starers}\PY{l+s+s2}{\PYZdq{}}\PY{p}{)}
\end{Verbatim}
\end{tcolorbox}

            \begin{tcolorbox}[breakable, size=fbox, boxrule=.5pt, pad at break*=1mm, opacityfill=0]
\prompt{Out}{outcolor}{37}{\boxspacing}
\begin{Verbatim}[commandchars=\\\{\}]
True
\end{Verbatim}
\end{tcolorbox}
        
    \hypertarget{exercise-10.7}{%
\subsection{Exercise 10.7}\label{exercise-10.7}}

Write a function called \texttt{has\_duplicates} that takes a list and
returns \texttt{True} if there is any element that appears more than
once. It should not modify the original list.

You can assume that the list will not have nested lists.

    \begin{tcolorbox}[breakable, size=fbox, boxrule=1pt, pad at break*=1mm,colback=cellbackground, colframe=cellborder]
\prompt{In}{incolor}{38}{\boxspacing}
\begin{Verbatim}[commandchars=\\\{\}]
\PY{k}{def} \PY{n+nf}{has\PYZus{}duplicates}\PY{p}{(}\PY{n}{t}\PY{p}{)}\PY{p}{:}
    \PY{k+kn}{import} \PY{n+nn}{numpy} \PY{k}{as} \PY{n+nn}{np}
    \PY{k}{if} \PY{n+nb}{len}\PY{p}{(}\PY{n}{np}\PY{o}{.}\PY{n}{unique}\PY{p}{(}\PY{n}{t}\PY{p}{)}\PY{p}{)} \PY{o}{!=} \PY{n+nb}{len}\PY{p}{(}\PY{n}{t}\PY{p}{)}\PY{p}{:}
        \PY{k}{return} \PY{k+kc}{True}
    \PY{k}{else}\PY{p}{:}
        \PY{k}{return} \PY{k+kc}{False}
\end{Verbatim}
\end{tcolorbox}

    \begin{tcolorbox}[breakable, size=fbox, boxrule=1pt, pad at break*=1mm,colback=cellbackground, colframe=cellborder]
\prompt{In}{incolor}{39}{\boxspacing}
\begin{Verbatim}[commandchars=\\\{\}]
\PY{n}{has\PYZus{}duplicates}\PY{p}{(}\PY{p}{[}\PY{l+s+s1}{\PYZsq{}}\PY{l+s+s1}{a}\PY{l+s+s1}{\PYZsq{}}\PY{p}{,}\PY{l+s+s1}{\PYZsq{}}\PY{l+s+s1}{b}\PY{l+s+s1}{\PYZsq{}}\PY{p}{,}\PY{l+s+s1}{\PYZsq{}}\PY{l+s+s1}{c}\PY{l+s+s1}{\PYZsq{}}\PY{p}{]}\PY{p}{)}
\end{Verbatim}
\end{tcolorbox}

            \begin{tcolorbox}[breakable, size=fbox, boxrule=.5pt, pad at break*=1mm, opacityfill=0]
\prompt{Out}{outcolor}{39}{\boxspacing}
\begin{Verbatim}[commandchars=\\\{\}]
False
\end{Verbatim}
\end{tcolorbox}
        
    \begin{tcolorbox}[breakable, size=fbox, boxrule=1pt, pad at break*=1mm,colback=cellbackground, colframe=cellborder]
\prompt{In}{incolor}{40}{\boxspacing}
\begin{Verbatim}[commandchars=\\\{\}]
\PY{n}{has\PYZus{}duplicates}\PY{p}{(}\PY{p}{[}\PY{l+s+s1}{\PYZsq{}}\PY{l+s+s1}{a}\PY{l+s+s1}{\PYZsq{}}\PY{p}{,}\PY{l+s+s1}{\PYZsq{}}\PY{l+s+s1}{b}\PY{l+s+s1}{\PYZsq{}}\PY{p}{,}\PY{l+s+s1}{\PYZsq{}}\PY{l+s+s1}{b}\PY{l+s+s1}{\PYZsq{}}\PY{p}{,}\PY{l+s+s1}{\PYZsq{}}\PY{l+s+s1}{c}\PY{l+s+s1}{\PYZsq{}}\PY{p}{]}\PY{p}{)}
\end{Verbatim}
\end{tcolorbox}

            \begin{tcolorbox}[breakable, size=fbox, boxrule=.5pt, pad at break*=1mm, opacityfill=0]
\prompt{Out}{outcolor}{40}{\boxspacing}
\begin{Verbatim}[commandchars=\\\{\}]
True
\end{Verbatim}
\end{tcolorbox}
        
    \begin{tcolorbox}[breakable, size=fbox, boxrule=1pt, pad at break*=1mm,colback=cellbackground, colframe=cellborder]
\prompt{In}{incolor}{41}{\boxspacing}
\begin{Verbatim}[commandchars=\\\{\}]
\PY{n}{has\PYZus{}duplicates}\PY{p}{(}\PY{p}{[}\PY{l+s+s1}{\PYZsq{}}\PY{l+s+s1}{a}\PY{l+s+s1}{\PYZsq{}}\PY{p}{,}\PY{l+s+s1}{\PYZsq{}}\PY{l+s+s1}{b}\PY{l+s+s1}{\PYZsq{}}\PY{p}{,}\PY{l+s+s1}{\PYZsq{}}\PY{l+s+s1}{c}\PY{l+s+s1}{\PYZsq{}}\PY{p}{,}\PY{l+s+s1}{\PYZsq{}}\PY{l+s+s1}{a}\PY{l+s+s1}{\PYZsq{}}\PY{p}{]}\PY{p}{)}
\end{Verbatim}
\end{tcolorbox}

            \begin{tcolorbox}[breakable, size=fbox, boxrule=.5pt, pad at break*=1mm, opacityfill=0]
\prompt{Out}{outcolor}{41}{\boxspacing}
\begin{Verbatim}[commandchars=\\\{\}]
True
\end{Verbatim}
\end{tcolorbox}
        
    \hypertarget{exercise-10.10}{%
\subsection{Exercise 10.10}\label{exercise-10.10}}

To check whether a word is in the word list, you could use the in
operator, but it would be slow because it searches through the words in
order.

Because the words are in alphabetical order, we can speed things up with
a bisection search (also known as binary search). You start in the
middle and check to see whether the word you are looking for comes
before the word in the middle of the list. If so, you search the first
half of the list the same way (perform a bisection search on the first
half). Otherwise you search the second half.

Either way, you cut the remaining search space in half. If the word list
has 113,809 words, it will take about 17 steps to find the word or
conclude that it's not there.

Write a function called \texttt{in\_bisect} that takes a sorted list and
a target word and will returns \texttt{True} if the word is in the list
and \texttt{False} if it's not.

Hint: it's a recursive function.

    \begin{tcolorbox}[breakable, size=fbox, boxrule=1pt, pad at break*=1mm,colback=cellbackground, colframe=cellborder]
\prompt{In}{incolor}{42}{\boxspacing}
\begin{Verbatim}[commandchars=\\\{\}]
\PY{c+c1}{\PYZsh{} Use this function. No need to rewrite it.}
\PY{k}{def} \PY{n+nf}{make\PYZus{}word\PYZus{}list}\PY{p}{(}\PY{p}{)}\PY{p}{:}
    \PY{l+s+sd}{\PYZdq{}\PYZdq{}\PYZdq{}Reads lines from a file and builds a list.\PYZdq{}\PYZdq{}\PYZdq{}}
    \PY{n}{t} \PY{o}{=} \PY{p}{[}\PY{p}{]}
    \PY{n}{fin} \PY{o}{=} \PY{n+nb}{open}\PY{p}{(}\PY{l+s+s1}{\PYZsq{}}\PY{l+s+s1}{words.txt}\PY{l+s+s1}{\PYZsq{}}\PY{p}{)}
    \PY{k}{for} \PY{n}{line} \PY{o+ow}{in} \PY{n}{fin}\PY{p}{:}
        \PY{n}{word} \PY{o}{=} \PY{n}{line}\PY{o}{.}\PY{n}{strip}\PY{p}{(}\PY{p}{)}
        \PY{n}{t}\PY{o}{.}\PY{n}{append}\PY{p}{(}\PY{n}{word}\PY{p}{)}
    \PY{k}{return} \PY{n}{t}

\PY{n}{t} \PY{o}{=} \PY{n}{make\PYZus{}word\PYZus{}list}\PY{p}{(}\PY{p}{)}
\end{Verbatim}
\end{tcolorbox}

    \begin{tcolorbox}[breakable, size=fbox, boxrule=1pt, pad at break*=1mm,colback=cellbackground, colframe=cellborder]
\prompt{In}{incolor}{43}{\boxspacing}
\begin{Verbatim}[commandchars=\\\{\}]
\PY{c+c1}{\PYZsh{} define this function}
\PY{k}{def} \PY{n+nf}{in\PYZus{}bisect}\PY{p}{(}\PY{n}{word\PYZus{}list}\PY{p}{,} \PY{n}{word}\PY{p}{)}\PY{p}{:}
    \PY{n}{list\PYZus{}mid} \PY{o}{=} \PY{n+nb}{int}\PY{p}{(}\PY{n+nb}{len}\PY{p}{(}\PY{n}{word\PYZus{}list}\PY{p}{)} \PY{o}{/} \PY{l+m+mi}{2}\PY{p}{)}
    \PY{k}{if} \PY{n}{word} \PY{o}{==} \PY{n}{word\PYZus{}list}\PY{p}{[}\PY{n}{list\PYZus{}mid}\PY{p}{]}\PY{p}{:}
        \PY{k}{return} \PY{k+kc}{True}
    \PY{k}{else}\PY{p}{:}
        \PY{k}{if} \PY{n}{list\PYZus{}mid} \PY{o}{==} \PY{l+m+mi}{0}\PY{p}{:}
            \PY{k}{return} \PY{k+kc}{False}
        \PY{k}{elif} \PY{n}{word} \PY{o}{\PYZlt{}} \PY{n}{word\PYZus{}list}\PY{p}{[}\PY{n}{list\PYZus{}mid}\PY{p}{]}\PY{p}{:}
            \PY{k}{return} \PY{n}{in\PYZus{}bisect}\PY{p}{(}\PY{n}{word\PYZus{}list}\PY{p}{[}\PY{p}{:}\PY{n}{list\PYZus{}mid}\PY{p}{]}\PY{p}{,} \PY{n}{word}\PY{p}{)}
        \PY{k}{else}\PY{p}{:}
            \PY{k}{return} \PY{n}{in\PYZus{}bisect}\PY{p}{(}\PY{n}{word\PYZus{}list}\PY{p}{[}\PY{n}{list\PYZus{}mid}\PY{p}{:}\PY{p}{]}\PY{p}{,} \PY{n}{word}\PY{p}{)}

        
        
\end{Verbatim}
\end{tcolorbox}

    \begin{tcolorbox}[breakable, size=fbox, boxrule=1pt, pad at break*=1mm,colback=cellbackground, colframe=cellborder]
\prompt{In}{incolor}{44}{\boxspacing}
\begin{Verbatim}[commandchars=\\\{\}]
\PY{n}{in\PYZus{}bisect}\PY{p}{(}\PY{n}{t}\PY{p}{,} \PY{l+s+s2}{\PYZdq{}}\PY{l+s+s2}{hello}\PY{l+s+s2}{\PYZdq{}}\PY{p}{)}
\end{Verbatim}
\end{tcolorbox}

            \begin{tcolorbox}[breakable, size=fbox, boxrule=.5pt, pad at break*=1mm, opacityfill=0]
\prompt{Out}{outcolor}{44}{\boxspacing}
\begin{Verbatim}[commandchars=\\\{\}]
True
\end{Verbatim}
\end{tcolorbox}
        
    \begin{tcolorbox}[breakable, size=fbox, boxrule=1pt, pad at break*=1mm,colback=cellbackground, colframe=cellborder]
\prompt{In}{incolor}{45}{\boxspacing}
\begin{Verbatim}[commandchars=\\\{\}]
\PY{n}{in\PYZus{}bisect}\PY{p}{(}\PY{n}{t}\PY{p}{,} \PY{l+s+s2}{\PYZdq{}}\PY{l+s+s2}{xyz}\PY{l+s+s2}{\PYZdq{}}\PY{p}{)}
\end{Verbatim}
\end{tcolorbox}

            \begin{tcolorbox}[breakable, size=fbox, boxrule=.5pt, pad at break*=1mm, opacityfill=0]
\prompt{Out}{outcolor}{45}{\boxspacing}
\begin{Verbatim}[commandchars=\\\{\}]
False
\end{Verbatim}
\end{tcolorbox}
        
    \hypertarget{exercise-10.11}{%
\subsection{Exercise 10.11}\label{exercise-10.11}}

Two words are a ``reverse pair'' if each is the reverse of the other.

Now that you have the \texttt{in\_bisect} search, write a script that
finds all the reverse pairs in the word list that are 6 letters or
longer. (It takes a little bit of time to run.)

    \begin{tcolorbox}[breakable, size=fbox, boxrule=1pt, pad at break*=1mm,colback=cellbackground, colframe=cellborder]
\prompt{In}{incolor}{46}{\boxspacing}
\begin{Verbatim}[commandchars=\\\{\}]
\PY{n+nb}{print}\PY{p}{(}\PY{l+s+s2}{\PYZdq{}}\PY{l+s+s2}{WORD}\PY{l+s+se}{\PYZbs{}t}\PY{l+s+s2}{REVERSE WORD}\PY{l+s+s2}{\PYZdq{}}\PY{p}{)}
\PY{k}{for} \PY{n}{word} \PY{o+ow}{in} \PY{n}{t}\PY{p}{:}
    \PY{n}{reverse\PYZus{}word} \PY{o}{=} \PY{l+s+s1}{\PYZsq{}}\PY{l+s+s1}{\PYZsq{}}
    \PY{k}{if} \PY{n+nb}{len}\PY{p}{(}\PY{n}{word}\PY{p}{)} \PY{o}{\PYZgt{}}\PY{o}{=} \PY{l+m+mi}{6}\PY{p}{:}
        \PY{n}{n} \PY{o}{=} \PY{n+nb}{len}\PY{p}{(}\PY{n}{word}\PY{p}{)}
        \PY{k}{for} \PY{n}{i} \PY{o+ow}{in} \PY{n+nb}{range}\PY{p}{(}\PY{n}{n}\PY{p}{)}\PY{p}{:}
            \PY{n}{reverse\PYZus{}word} \PY{o}{=} \PY{n}{reverse\PYZus{}word} \PY{o}{+} \PY{n}{word}\PY{p}{[}\PY{o}{\PYZhy{}}\PY{l+m+mi}{1} \PY{o}{\PYZhy{}} \PY{n}{i}\PY{p}{]}
        \PY{k}{if}  \PY{n}{in\PYZus{}bisect}\PY{p}{(}\PY{n}{t}\PY{p}{,} \PY{n}{reverse\PYZus{}word}\PY{p}{)}\PY{p}{:}
            \PY{n+nb}{print}\PY{p}{(}\PY{n}{word} \PY{o}{+} \PY{l+s+s2}{\PYZdq{}}\PY{l+s+se}{\PYZbs{}t}\PY{l+s+s2}{\PYZdq{}} \PY{o}{+} \PY{n}{reverse\PYZus{}word}\PY{p}{)}
\PY{n+nb}{print}\PY{p}{(}\PY{l+s+s2}{\PYZdq{}}\PY{l+s+s2}{Abovt is all reverse pair from word list}\PY{l+s+s2}{\PYZdq{}}\PY{p}{)}
\end{Verbatim}
\end{tcolorbox}

    \begin{Verbatim}[commandchars=\\\{\}]
WORD    REVERSE WORD
agenes  senega
animal  lamina
animes  semina
degami  imaged
deified deified
deifier reified
deliver reviled
denier  reined
denies  seined
denned  denned
depots  stoped
derats  stared
dessert tressed
desserts        stressed
dewans  snawed
dialer  relaid
diaper  repaid
dormin  nimrod
drawer  reward
elides  sedile
eviler  relive
gelder  redleg
halalah halalah
hallah  hallah
imaged  degami
lamina  animal
levins  snivel
looter  retool
marram  marram
nimrod  dormin
pupils  slipup
recaps  spacer
redder  redder
redips  spider
redleg  gelder
redraw  warder
redrawer        rewarder
reflet  telfer
reflow  wolfer
reified deifier
reifier reifier
reined  denier
reknit  tinker
reknits stinker
relaid  dialer
relive  eviler
remeet  teemer
rennet  tenner
repaid  diaper
repaper repaper
repins  sniper
retool  looter
reviled deliver
reviver reviver
reward  drawer
rewarder        redrawer
rotator rotator
sallets stellas
scares  seracs
secret  terces
sedile  elides
seined  denies
selahs  shales
selles  selles
sememes sememes
semina  animes
senega  agenes
seracs  scares
shales  selahs
skeets  steeks
sleeps  speels
sleets  steels
slipup  pupils
sloops  spools
snawed  dewans
sniper  repins
snivel  levins
snoops  spoons
spacer  recaps
speels  sleeps
spider  redips
spirts  strips
spools  sloops
spoons  snoops
sports  strops
sprits  stirps
stared  derats
steeks  skeets
steels  sleets
stellas sallets
stinker reknits
stirps  sprits
stoped  depots
stressed        desserts
strips  spirts
strops  sports
struts  sturts
sturts  struts
teemer  remeet
telfer  reflet
tenner  rennet
terces  secret
terret  terret
tinker  reknit
tressed dessert
warder  redraw
wolfer  reflow
Abovt is all reverse pair from word list
    \end{Verbatim}


    % Add a bibliography block to the postdoc
    
    
    
\end{document}
